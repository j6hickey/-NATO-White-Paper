
NATO Science & Technology Organization (STO)
Collaboration Support Office (CSO)
Applied Vehicle Technology Panel (AVT)

Preliminary Meeting Announcement and 
Call for Papers
AVT-SET-396 Research Specialists’ Meeting (tbc)
on
Technological Challenges for Hypersonic Flights (tbc)

organised by the Members of the 
Applied Vehicle Technology Panel, Sensors & Electronics Technology Panel and Systems Concepts and Integration Panel
AVT-SET-396  (+SCI) Programme Committee
to be held in tbc, Germany
X-Y October 2024 tbc


Contributions and participation are invited from NATO Nations
plus Australia, Sweden and Japan

Changes to the event title and the activity type (to a Research Symposium) are pending approval by the STB in late Fall 2023

Note: Final date for submission of abstracts is 15 November 2023



AVT-SET-396 Specialists’ Meeting (tbc) on
Technological Challenges for Hypersonic Flights (tbc)

Framework of the Meeting
The Applied Vehicle Technology (AVT), Sensors & Electronics Technology (SET) and Systems Concepts and Integration (SCI) Panels of the NATO Science and Technology Organization (STO) are organizing a Cross Panel Research Specialists’ Meeting (RSM) on the subject of “Technological Challenges for Hypersonic Flights 

The meeting is open to NATO Nations plus the “Enhanced Opportunity Partner” nations Australia, Sweden and Japan, and is classified as “NATO UNCLASSIFIED OPEN TO AUSTRALIA, SWEDEN and JAPAN”. The Meeting is to be held in tbd, Germany, from X-Y October 2024.
 
An RSM is a four-day event with possibly more than 100 participants that aims at presenting and promoting exchange of state-of-the-art knowledge among an audience of experts and specialists on an important scientific or applied topic. The Programme Committee is responsible for selecting and inviting the Speakers. Observers, who present no paper, can of course also participate. RSM’s result in an STO Report (Meeting Proceedings). 

Selected papers from the RSM will be considered for publication in the NATO STO Review Journal after an extensive technical peer review process. The authors of the subset of papers put forward for the peer review process and journal publication will be contacted after the RSMhas taken place.

There will also be a Best Paper Award. The winner will be announced and awarded during the summary session of the RSM in tbd, Germany.





General Scope and Meeting Objectives
Hypersonic flight is defined as flight in the higher atmosphere at a speed of more than Mach five (five times the speed of sound) and with the capability to do deceptive and evasive manoeuvres during flight. Hypersonics was identified as an Emerging Disruptive Technology (EDT) by NATO. It is a game-changer technology with potential benefits including increased survivability, high effectiveness against time-critical targets and strategic deterrence. Defence against hypersonic threats such as Hypersonic Glide Vehicles (HGV), Hypersonic Cruise Missiles (HCM) and Hypersonic Rockets (HR) is challenging beyond current Integrated Air and Missile Defence capabilities.
Potential opposing Nations like Russia and China have fielded operational hypersonic weapons and Russia has used the ‘Kinzhal’ (Dagger) hypersonic missile in the Ukraine war. NATO Nations as well as Australia and Japan are substantially investing in hypersonic research and on building operational capabilities and defence.

NATO-STO AVT Specialists Team 008 (AVT-ST-008) has assessed Status and Challenges posed by Hypersonic Operational Threats (Report TR-AVT-ST-008) together with a comprehensive Technical and Scientific Annex dealing with key and enabling technologies for hypersonic flight and related vehicles.
				
NATO-STO AVT Long Term Scientific Study (AVT-359-LTSS) investigates Impact of Hypersonic Operational Threats on Military Operations and Technical High Level Requirements. The reports of Phase 1 (TR-AVT-359-Part 1) and of Phase 2 (report in progress) assess extensively the phases of the OODA (Observe-Orient-Decide-Act) Loop.
NATO-STO Research Workshop AVT-SCI-379-RWS) on Technology Needs for Hypersonic Operational Threats (TecNHOT) invited all STO Panels to discuss issues within their portfolio posed by Hypersonic Operational Threat and to propose new activities.
NATO-STO SET Research Task Group SET-296 RADAR against Hypersonic Threat currently investigates the abilities of different RADAR systems for detecting and tracking hypersonic weapon systems and related technical issues.
NATO-STO AVT Research Task Group AVT-376 Methodology for Tactical Missile IR Signature Predictions, develop strategies to accurately predict propulsion and aerothermal signatures of advanced military vehicles, what will partly apply to hypersonic vehicles for the launch/boost phase and HCM propulsion.
A new activity (AVT-RTG-P 2023-06) is being proposed to generate a generic aggregated design of hypersonic gliding vehicles (HGV).
The NATO-STO SCI-Panel collaborates with these activities in the field of systems analysis, command and control and defence architectures for strike and defence, and related technologies.
A new cross-panel Research Task Group NMSG-AVT-SCI (NMSG: NATO Modelling and Simulation Group) on a Common Synthetic Environment for Validation (or operation) of Offensive and Defensive Architectures for Hypersonic Operational Threats (HOT) is being proposed to develop and provide government, military and industrial leaders and technical staff in NATO nations with a common synthetic environment to investigate and understand the HOT domain. 
It appears timely to present the findings and conclusions of these activities to military operators and decision makers and to discuss these with the expert community. The objective of the RSM s to distribute information to understand issues of hypersonic flight vehicles, to provide a realistic appraisal of hypersonic weapons feasibility and capabilities, and to also help to demystify the potential threat.

RSM Topics
Papers are invited in the areas of: 
•	Aerothermodynamics and effects
•	Design, Structures and Materials for hypersonic vehicles (incl. material characterization at high temperatures)
•	Propulsion technologies (RamJets, SCRamJets, Detonation Engines and rockets)
•	Platform Technologies (HGV, HCM, HR, Interceptors)
•	System Architecture Challenges (HGV, HCM, HR, Interceptors) – military requirements vs. technical needs and feasibility
•	Launch Platforms - Options and Constraints
•	Guidance and Control, Communication (strike and defensive)
•	Onboard sensors and issues (strike and defensive)
•	Detecting and tracking hypersonic weapons (RADAR, IR, related Issues)
•	Defence Architecture Considerations (Defence design, Area Defence, Point Defence, synthetic environment …)
•	C3I (Command, Control, Communication and Intelligence) for hypersonic weapons defence
•	Conceptual Design Examples for offence (HGV,HCM, HR)
•	Conceptual Design Examples for defence (Interceptors)
•	Non-Kinetic Defence against hypersonic Weapons (Laser, Directed Energy, Blinding, Jamming …)
The topics below apply to all preceding topics
•	Modelling and Simulation Challenges and Results (e.g. trajectories, manoeuvrability, terminal aproach for strike and intercept, vulnerability/lethality, impact, …)
•	Ground Testing and Experimentation
•	Flight Testing

Background and Justification - Relevance to NATO
The hypersonic weapons threat has emerged in recent years and operational capabilities were claimed and tested by RED Nations. The Air-Launched hypersonic missile Kinzhal is being used by Russia in the Ukraine war with limited success and in some cases, there may have been an intercept by Ukrainian air defence.

Research and development work is urgent and ongoing in NATO and affiliate Nations to build up operational capabilities for hypersonic strike and for defence against hypersonic weapons threat. For this it is paramount to understand challenges posed by hypersonic flight and design and architecture of hypersonic flight vehicles. It is equally important to understand capabilities and limitations of hypersonic weapons to design effective defence systems reaching from general architecture to the individual components for detecting, tracking and intercept.

Together with reports originated from the NATO STO activities cited before (cf. pg. 3) this Specialists’ Meeting will summarize state of the art technology and important fields of research to fill remaining gaps. 

Development work run by NATO Nations towards operational hypersonic for strike or defence or classified intelligence will not be considered due to the ‘NU’- format of the Specialists’ Meeting. However, a broad technological overview will pinpoint key feasibility issues for hypersonic flight and related vehicles and assessment of generic weapon systems will provide important information about inherent capabilities and limitations.


 

AVT-SET-396 Technical Programme Committee 
Co-Chairs

for AVT Panel
Hans-Ludwig Besser
German Aerospace Centre DLR-RSC3
Argelsrieder Feld 1A
82234 Weßling, Germany
Tel.: +49 151 17434304 
Email: hans-ludwig.besser@dlr.de

for SET Panel
Daniel O’Hagan
Fraunhofer Gesellschaft
Fraunhoferstrasse 20
5333 Wachtberg, Germany
Tel: +49 228 9435389 
Email: daniel.ohagan@fhr.fraunhofer.de

for SCI Panel
Rogerio Pimentel
Defence Research and Development Canada
Valcartier Research Centre
2459 de la Bravoure Road,
Quebec, Canada, G3J1X5
Tel : +1 418 844 4000 4170
			     Email: Rogerio.GomesPimentel@forces.gc.ca

Canada
Prakash Patnaik
Aerospace Research Centre
National Research Council Canada
1200 Montreal Road, Building M-17
Ottawa, Ontario, Canada, K1A 0R6
Tel: +1 613-355-5099
Email: Prakash.Patnaik@nrc-cnrc.gc.ca

Allison Nolting
Defence Research and Development Canada
Atlantic Research Centre
PO Box 99000 Station Forces,
Dartmouth, Nova Scotia, Canada, B3K 5X5
Tel: +1 902 407 0387
Email: Allison.Nolting@forces.gc.ca
	Netherlands
Linda van der Ham
TNO - Defence Safety and Security…
Oude Waalsdorperweg 63, 
2597 AK Den Haag / The Hague, Netherlands
Tel: +31 88 866 22 23
Email: Linda.Vanderham@tno.nl

USA
Russell Cummings
USAF Academy 
2354 Fairchild Dr., Suite 6H-101, HQ USAFA/DFAN, USAF Academy, CO 80840
USA
Tel: +17193069827
Email: Russell.Cummings@afacademy.af.edu

Deadlines and Preliminary Schedule
14 Aug 2023
	Distribution of Call for Papers
To solicit abstracts from NATO nations and all other included Nations
After: authors to send their abstracts to the Programme Committee

15 Nov 2023	Abstract Submission Deadline 
After: Programme Committee to select abstracts and to create the Meeting Programme from selected abstracts

15 Feb 2024	Authors Informed of Selection Decision
Programme Committee to inform selected as well as rejected authors
AVT to dispatch authors’ information package to selected authors
After: selected authors to prepare their papers, presentation and clearances

5 Apr 2024	Final Agenda Approved by the Programme Committee
Programme Committee to finalize the Programme
After: AVT to prepare and publish the official Meeting Announcement

14 Jun 2024	Submission of Advance Copy of US Presentations/Papers to US National Coordinator
Deadline for US authors to submit a copy of their advance paper to the US National Coordinator (special instructions to be issued with authors’ information package)

15 Jul 2024	Electronic Advance Copy of Presentations/Paper due at AVT
Deadline for all other authors to send an advance copy of their full scientific
Paper to AVT
After: Technical Evaluator to review all submitted papers

6 Sep 2024	Submission of Final Version of all Presentations/Papers to AVT
Deadline for all authors to send final cleared papers to AVT
After: AVT to pre-release all papers on the STO website making them 
accessible to all registered participants of the Specialists‘ Meeting

X-Y Oct 2024	Specialists’ Meeting to be held in tbd, Germany
	
15 Nov 2024	Submission of Corrected Manuscripts
Deadline for all presentations/papers to be included in the Meeting Proceedings
After: AVT to edit, prepare, produce Meeting Proceedings, which will be made accessible through the STO website
 
Procedures
Invitation and Abstract Submission
The initial abstract should describe in 1000-1500 words the aim, results and conclusions of the work. Inclusion of 1 to 2 figures and/or photographs to support the abstract is recommended. Authors’ names, complete email addresses and other pertinent information must be included with the abstracts. For this purpose, please use the Abstract Submittal Form (Annex 1) and keep the size of files less than 2 MB. 
The program committee will select the abstracts for either a presentation or a presentation plus full scientific paper. The abstracts and papers will be included in the meeting proceedings. 
The full scientific paper (recommended limit of 12 pages) will be requested once the Programme Committee has selected the authors and developed the final agenda for the Meeting.

Please submit your abstract along with the Abstract Submittal Form by no later than 15 Nov. 2023 via email to the Programme Committee Co-Chairs (contact data on pg. 6).


Security Level, Clearance and Paper Preparation
This Specialists’ Meeting is classified as NATO UNCLASSIFIED OPEN TO AUSTRALIA, SWEDEN and JAPAN. For details please consult the attached section on NATO and Partner Nations Overview (Annex 2). It is the responsibility of each contributor to fulfil the publication release and clearance requirements of his/her organization/company/affiliation and country to obtain clearance of abstracts and papers as needed. An official clearance is mandatory in the US (see Annex 3) and there may also be a requirement in other countries. If in doubt, authors should contact a Programme Committee Member.

Authors of papers selected for presentation and publication will be notified by the Programme Committee by no later than 15 February 2024. The AVT Executive Office will then send an Authors’ Information Package containing templates, detailed instructions concerning the preparation of manuscripts, as well as, information about the clearance process to each lead author.



Travel and Logistics
Authors of contributions selected for presentation will not be financially supported by NATO. Authors are responsible for their own hotel and travel reservations based on suggestions given in the General Information Package which will be provided typically 4 months ahead of time. Expenses for travel and per diem costs are the responsibility of each author’s organisation and nation.


Contact Information
Any questions about the technical aspects of the scientific programme or the contents of papers should be addressed to the Programme Committee Co-Chairs listed above.

Questions on the administrative aspects of this Research Specialists’ Meeting or requests for further information about STO activities should be addressed to:

Ms. Isavela Kontolaimaki
AVT Panel Executive Assistant
AVT Executive Office
NATO/STO Collaboration Support Office
BP 25 - F-92201 Neuilly-sur-Seine, Cedex 01 - France
Tel: + 33(0)1 55 61 22 88
Email: isavela.kontolaimaki@cso.nato.int



Thank you for your contributions 
which are very much appreciated by the NATO community.
 

The NATO Science & Technology Organization
The NATO Science & Technology Organization promotes and conducts co-operative scientific research and exchange of technical information amongst NATO nations and NATO partners. Being the largest such collaborative body in the world, the STO encompasses over 5000 scientists and engineers addressing the complete scope of defence technologies and operational domains. This effort is supported by the Collaboration Support Office, which facilitates the collaboration by organising a wide range of studies, workshops, symposia, and other fora in which researchers can meet and exchange knowledge. 

For further information, please consult the STO web site: www.sto.nato.int

The STO website provides a wide variety of information and on-line services ranging from overview information on the organization’s mission to news regarding upcoming events. You will find on-line access to more than 1800 scientific publications, as well as, information about current activities. 


Applied Vehicle Technology
The Applied Vehicle Technology Panel, comprising more than 1000 scientists and engineers, strives to improve the performance, reliability, affordability, and safety of vehicles through advancement of appropriate technologies. The Panel addresses platform technologies for vehicles operating in all domains (land, sea, air, and space), for both new and ageing systems.
The members of the AVT community exploit and focus their joint expertise in the following fields: 

•	Mechanical Systems, Structures, and Materials
•	Performance, Stability and Control, Fluid Physics 
•	Propulsion and Power Systems 


For further information please consult the AVT web site.



Sensors & Electronics Technology
The Sensors & Electronics Technology (SET) Panel is eager to advance technology in electronics and passive/active sensors (as they pertain to reconnaissance, surveillance, target acquisition, electronic warfare, communications, navigation) and to enhance sensor capabilities through multi-sensor integration/fusion in order to improve the operating capability and to contribute to fulfil strategic military results. As NATO war-fighters and peace-keepers continue to shift more and more towards asymmetrical warfare, SET technology have to focus on the military mission of saving lives, improving quality of life and extending our combat effectiveness. Research in the Sensors and Electronics Technology Panel concerns the phenomenology related to target signature, propagation and battle-space environment, electro-optics (or electro-optical, EO), radio frequency (RF), acoustic and magnetic sensors, antenna, signal and image processing, components, sensor hardening and electromagnetic compatibility.
For further information please consult the SET web site

Systems Concepts and Integration
The Systems Concepts and Integration Panel’s mission is to advance knowledge concerning the integration of people, equipment and information systems that form a capability or capabilities in support of prosecuting a mission. This includes advanced systems and concepts, integration and systems engineering techniques, and cross-cutting technologies that assure delivery of effective and efficient capabilities. In addition, since NATO operates as a coalition, issues related to systems of systems are pre-eminent for the SCI Panel, emphasizing architecture, interoperability, and compatibility.
SCI Panel consist of the following Working Sessions:
•	Systems Integration and Interoperability
•	Integrated Survivability
•	Enablers and Disruptive Capabilities

For further information please consult the SCI web site.

 
Annex 1
Abstract Submittal Form
AVT-SET-396 Specialists’ Meeting on “Technological Challenges for Hypersonic Flights” 

Please attach a copy of this form to each abstract.



TITLE OF PAPER: 





1. Author Title			Name						Nationality

____________________________________________________________________________
Affiliation: ____________________________________________________________________________
Full Mailing Address: ____________________________________________________________________________

____________________________________________________________________________
Telephone / Email address: ____________________________________________________________________________



2. Author Title			Name						Nationality

____________________________________________________________________________
Affiliation: ____________________________________________________________________________
Full Mailing Address: ____________________________________________________________________________

____________________________________________________________________________
Telephone / Email address: ____________________________________________________________________________


3. Author Title			Name						Nationality

____________________________________________________________________________
Affiliation: ____________________________________________________________________________
Full Mailing Address: ____________________________________________________________________________

____________________________________________________________________________
Telephone / Email address: ____________________________________________________________________________


3. Author Title			Name						Nationality

____________________________________________________________________________
Affiliation: ____________________________________________________________________________
Full Mailing Address: ____________________________________________________________________________

____________________________________________________________________________
Telephone / Email address: ____________________________________________________________________________

Please copy as required – recommended not more than 5 authors.



Note bene:
Authors should be listed in the order they will appear on the programme and in the final manuscript. Unless specified otherwise, the first listed author is presumed to be the lead author having the major responsibility regarding content of the paper. 
 
Annex 2

NATO and Partner Nations Overview
 with Geographical Abbreviations

NATO NATIONS
ALBANIA	ALB	LITHUANIA	LTU
BELGIUM	BEL	LUXEMBOURG	LUX
BULGARIA	BGR	MONTENEGRO	MNE
CANADA	CAN	NORTH MACEDONIA	MKD
CROATIA	HRV	NORWAY	NOR
CZECH REPUBLIC	CZE	POLAND	POL
DENMARK	DNK	PORTUGAL	PRT
ESTONIA	EST	ROMANIA	ROU
FRANCE	FRA	SLOVAKIA	SVK
FINLAND	FIN	SLOVENIA	SVN
GERMANY	DEU	SPAIN	ESP
GREECE	GRC	THE NETHERLANDS	NLD
HUNGARY	HUN	TURKEY	TUR
ICELAND	ISL	UNITED KINGDOM	GBR
ITALY	ITA	UNITED STATES	USA
LATVIA	LVA		

EAPC/PARTNERSHIP for PEACE NATION (PfP)
ARMENIA	ARM	MALTA	MLT
AUSTRIA	AUT	MOLDOVA	MDA
AZERBAIJAN	AZE	SERBIA	SRB
BELARUS	BLR	SWEDEN*	SWE
BOSNIA & HERZEGOVINA	BIH	SWITZERLAND	CHE
MACEDONIA	MKD	TAJIKISTAN	TJK
GEORGIA	GEO	TURKMENISTAN	TKM
IRELAND	IRL	UKRAINE	UKR
KAZAKHSTAN	KAZ	UZBEKISTAN	UZB
KYRGYZSTAN	KGZ		





MEDITERRANEAN DIALOGUE NATION (MD)
ALGERIA	DZA	MAURITANIA	MRT
EGYPT	EGY	MOROCCO	MAR
ISRAEL	ISR	TUNISIA	TUN
JORDAN	JOR		

ISTANBUL COOPERATION INITIATIVE (ICI) NATION LIST
BAHRAIN	BHR	SAUDI ARABIA	SAU
KUWAIT	KWT	UNITED ARAB EMIRATES	ARE
QATAR	QAT		

GLOBAL PARTNERS (GP)
AFGHANISTAN	AFG	KOREA (Republic Of)	KOR
AUSTRALIA*	AUS	NEW ZEALAND	NZL
IRAQ	IRQ	PAKISTAN	PAK
JAPAN	JPN		

OTHER NATIONS (including CONTACT COUNTRIES)
ARGENTINA	ARG	CHINA	CHN
BRAZIL	BRA	INDIA	IND
CHILE	CHL	SINGAPORE	SGP


* Australia, Sweden, and Japan are usually referred to as “Enhanced Opportunity Partner” (EOP).
 
Annex 3

Special Notice for U.S. Authors
and non U.S. Authors Affiliated with U.S. Organizations
Abstracts of Papers from the U.S. must be sent to the following P.O.C.
 4 weeks prior to the regular abstract submission deadline:
NATO S&T Organization U.S. National Coordinator
OASD(R&E)/International Technology Programs
4800 Mark Center Drive, Suite 17D08
Alexandria, VA 22350-3600
Tel: +1 571-372-6539 / 6538
E-mail: osd.pentagon.ousd-atl.mbx.usnatcor@mail.mil
In addition to their abstract, all U.S. Authors must provide: 
1. Certification (can be signed by the author) that there are no proprietary or copyright limitations; 
2. Internal documentation from their local public affairs or foreign disclosure office (or equivalent) that clearly shows:
- Title of the paper or presentation 
- Level of clearance (i.e., Approved for Public Release) 
- Name, title, and organization of the approval authority 
3. Full details of authors 

Note that only complete packages (abstracts + items listed above) will be accepted by the U.S. PoC.

After review and approval, the U.S. PoC will forward all U.S. abstracts to the AVT Panel Office, who will send them to the Programme Committee. 

U.S. authors are encouraged to address questions and concerns to the PoC as early as possible to the US Member of the Programme Committee. Delays in meeting deadlines will impact the timely submission of your abstract.


